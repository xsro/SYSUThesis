% 作者:1FCENdoge
%%
% 下面两行代码2选1,分别适用于overleaf环境下的编译和windows下的编译
\documentclass{SysuthesisOverleaf}
%\documentclass{SysuthesisWindows}

\usepackage{sysucode}  % 在论文中使用代码

%%
% 论文相关信息
% 本文档中前缀"c-"代表中文版字段, 前缀"e-"代表英文版字段
%%

% 扉页标题
% 论文题目应以简短、明确的词语恰当概括整个论文的核心内容,避免使用不常见的缩略词、缩写字。读者通过标题可大致了解毕业设计(论文)的内容、专业的特点和科学的范畴。中文题目一般不宜超过25个字。
\covertitlefirst{中山大学硕士毕业论文模板(试用)}
\covertitlesecond{\LaTeX\ template for undergraduate thesis of SYSU}
% 第二行英文标题,英文标题过长时使用,不用时留空即可,删除会导致编译问题
\covertitlethird{}

% 摘要页中文标题
\ctitle{中山大学硕士毕业论文模板(试用)}
% 摘要页英文标题
\etitle{\LaTeX\ template for undergraduate thesis of SYSU}

% 解决英文摘要页标题过长问题 (Issue 49&63)
% 当\etitle的长度超过页边距时,请使用下面的命令自行断行
% 此操作只影响英文摘要页的标题,不影响页眉的标题
% 如果不需要断行,将\eabstracttitlesecond{ }留空即可
%\eabstracttitlefirst{} 
%\eabstracttitlesecond{}

% 作者(扉页、摘要页)
\cauthor{黄小明}
\eauthor{HUANG Xiaoming}

% 专业(扉页、摘要页)
\cmajor{核技术及应用}
\emajor{Nuclear Technology and Application}

% 指导老师(扉页、摘要页)
\cmentor{王大明(副教授)}
\ementor{Assoc. Prof. WANG Daming}

     % 论文相关信息
%%
% 摘要
%%
% 中文摘要
\cabstract{

摘要内容应概括地反映出本论文的主要内容,主要说明本论文的研究目的、内容、方法、成果和结论。要突出本论文的创造性成果或新见解,不要与引言相混淆。语言力求精练、准确,硕士论文摘要以800-1200字为宜。在摘要的下方另起一行,注明本文的关键词(3-5个)。关键词是供检索用的主题词条,应采用能覆盖论文主要内容的通用技术词条(参照相应的技术术语标准)。按词条的外延层次排列,外延大的排在前面,每个关键词用“;”分开,最后一个关键词不打标点符号。}
% 中文关键词
\ckeywords{硕士学位论文;\LaTeX\ 模板;中山大学}

% 英文摘要
\eabstract{

英文摘要内容与中文摘要相同。摘要下方另起一行注明英文关键词(Keywords 3-5个),每个关键词之间用英文分号加空格分开,最后一个关键词不打标点符号。
}
% 英文关键词
\ekeywords{undergraduate thesis; \LaTeX\ template; Sun Yat-sen University}

     % 摘要内容
\begin{document}
% 论文前置部分
\frontmatter
\pagenumbering{Roman}
\maketitle    % 封面
% 定稿打印的论文需在扉页、学术诚信声明后添加空白页以保证这两部分内容各自独占一张纸,同时根据情况添加空白页以保证正文第一页为奇数页面,送盲审论文则可以不用添加空白页。
% 双数页面为空白页
\cleardoublepage
\makedisclaim    % 学术诚信声明
\cleardoublepage
\makeabstract       % 中英文摘要
\maketableofcontents        % 目录
\cleardoublepage

% 论文主体部分
\mainmatter
% 正文
% 硕士学位论文一般有5到6章,博士学位论文一般有6到7章
%%
% 引言或背景
% 引言是论文正文的开端,应包括毕业论文选题的背景、目的和意义;对国内外研究现状和相关领域中已有的研究成果的简要评述;介绍本项研究工作研究设想、研究方法或实验设计、理论依据或实验基础;涉及范围和预期结果等。要求言简意赅,注意不要与摘要雷同或成为摘要的注解。
%%
% 章、节、小节、图片、公式、表格下方的\label{...}标记不建议删除,因为这些可以做到自动引用的作用,当某些公式、图片被删除时,\label{...}标记能使正文中的编号自动更新,省去一个一个编号的麻烦。

\chapter{绪论}
\label{cha:introduction}
\section{引言}
\label{sec:prologue}
引言是论文正文的开端,应包括毕业论文选题的背景、目的和意义;对国内外研究现状和相关领域中已有的研究成果的简要评述;介绍本项研究工作研究设想、研究方法或实验设计、理论依据或实验基础;涉及范围和预期结果等。要求言简意赅,注意不要与摘要雷同或成为摘要的注解。

\section{国内外研究现状和相关工作}
\label{sec:related_work}
对国内外研究现状和相关领域中已有的研究成果的简要评述。

\section{快速上手}
\label{sec:latex_basic}
本模板不会提及过多花里胡哨的操作,只追求使用者正确配置、快速上手使用LaTeX,拿到模板后即能专注于论文内容的撰写,不会纠结于配置以及其它有关代码的问题。下面介绍一种能成功配置的方法,是我使用的配置方法。

以下操作步骤均在Windows 10/11操作系统中完成,不建议在Linux系统中操作,因为Linux系统会涉及额外的字体安装、配置等问题,并且本人实测,同样是texlive编译,同样的文件,同一台电脑安装Ubuntu 20.04.3/Windows 11双系统,在Windows 10/11中编译生成pdf更快。

第一步,下载texlive并安装。建议从知名的开源镜像站下载安装包,如中科大开源镜像站(图 \ref{fig:ustc})、清华大学开源镜像站(图 \ref{fig:tsinghua})等,texlive一般在CTAN目录下(图 \ref{fig:texlive}),建议下载并安装2021及更新版本。
\begin{figure}[htbp] 
	\centering
	\includegraphics[width=0.85\textwidth]{image/chap01/ustc.png}
	\caption{中科大开源镜像站}
	\label{fig:ustc}
\end{figure}
\begin{figure}[htbp] 
	\centering
	\includegraphics[width=0.85\textwidth]{image/chap01/tsinghua.png}
	\caption{清华大学开源镜像站}
	\label{fig:tsinghua}
\end{figure}
\begin{figure}[htbp] 
	\centering
	\includegraphics[width=0.7\textwidth]{image/chap01/texlive.png}
	\caption{TeX Live下载}
	\label{fig:texlive}
\end{figure}

第二步,安装CTAN宏包。在开始菜单中找到Tex Live command-line(图 \ref{fig:texlivecmd}),以管理员模式运行,依次运行以下两行命令:

\noindent tlmgr option repository http://mirrors.aliyun.com/CTAN/systems/texlive/tlnet/

\noindent tlmgr update $ \text{-}\text{-} $self $ \text{-}\text{-} $all

\noindent 等待CTAN宏包更新自动完成即可。
\begin{figure}[htbp] 
	\centering
	\includegraphics[width=0.6\textwidth]{image/chap01/texlivecmd.png}
	\caption{Tex Live command-line}
	\label{fig:texlivecmd}
\end{figure}

第三步,安装TeXstudio并配置。一般来说安装完成之后TeXstudio能自动识别已安装的texlive,打开TeXstudio,在上方找到Options $\rightarrow$ Configure TeXstudio并点击(图 \ref{fig:texstudio1}),在Build中将Default Compiler选为XeLaTeX,Default Bibliography Tool(默认参考文献工具)选为BibTeX(图 \ref{fig:texstudio2})。
\begin{figure}[htbp] 
	\centering
	\includegraphics[width=1\textwidth]{image/chap01/texstudio1.png}
	\caption{TeXstudio Configure位置}
	\label{fig:texstudio1}
\end{figure}
\begin{figure}[htbp] 
	\centering
	\includegraphics[width=1\textwidth]{image/chap01/texstudio2.png}
	\caption{TeXstudio Configure设置}
	\label{fig:texstudio2}
\end{figure}

第四步,编译生成pdf文档,操作为:用TeXstudio打开main.tex文件,点击上方绿色双箭头(Build \& View)(图 \ref{fig:f5}),等待LaTeX自动完成编译过程,就能生成正确的pdf文档。
\input{figure/chap01/f5}

综上,上述四个步骤的操作能让你在Windows 10/11+texlive+TeXstudio的环境下得到与github页面中内容一模一样的pdf文档。

\section{本文的论文结构与章节安排}
\label{sec:arrangement}
本文共分为五章,各章节内容安排如下:

第一章为绪论。

第二章为本模板遵循的排版及格式。

第三章为图像的插入示例。

第四章为公式与表格的插入示例

第五章是本文的最后一章,结论与展望。是对本文内容的整体性总结以及对未来工作的展望。


\chapter{本模板遵循的排版及格式}
\label{cha:format}
\chapter{图像的插入示例}
\label{cha:fig_example}
除了第一章引言和最后一章的总结与展望之外,正文的所有章都要在章标题之下加上这样一段引入本章内容的话语,让读者知道本章的目的以及意义。本章将通过一些示例来说明如何插入图片。读者在阅读文章时,最能吸引读者注意力的莫过于文章中的图片,因此图片对于论文来说是重中之重,甚至可以说,好图就是好文章。规范地插入图片对于整篇文章的观感、阅读体验来说,有着至关重要的作用。
\section{单张图片的插入}
\label{sec:fig_singlefig}
单张图片插入的原则:(1)图片居中放置,大小适当,图中文字、内容清晰;(2)从文献中获得的图片要引用,要写明来源;(3)图片应该放置在两段文字之间,图片上面一段文字应该是对图片内容的描述,不要插在一段文字内,一页排不下时,应排在下一页的顶部;(4)对图片的描述要符合规范,指明是图x-x,不能说如下图所示。
\begin{itemize}
\item 错误描述:托卡马克装置示意图如下图所示\cite{xu2016general}:
\item 正确描述:托卡马克装置示意图如图 \ref{fig:tokamak} 所示\cite{xu2016general}:
\end{itemize}

\begin{figure}[htbp] % 图片排序优先级,h表示当前位置,t表示顶部,b表示底部,p表示浮动页,可以是单独一个字母或者几个字母的组合
	% 居中
	\centering
	% 图片宽度、图片文件名及在硬盘中的位置
	\includegraphics[width=0.6\textwidth]{image/chap03/tokamak.png}
	% 图片下标题
	\caption{托卡马克装置示意图\cite{xu2016general}}
	% 图片标签
	\label{fig:tokamak}
\end{figure}
\subsection{矢量图片的插入}
\label{ssec:fig_vecfig}
本小节示例了如何插入小节。按照中大的规定,正文中的标题只到小节,如 \ref{ssec:fig_vecfig} 小节,目录中的标题只到节,如 \ref{sec:fig_singlefig} 节。

\LaTeX\  支持svg、pdf、eps格式的矢量图的插入,svg格式的矢量图插入过程有点复杂,我暂时还没看明白,但是pdf和eps格式的矢量图是能直接插入的,操作很简单,与图 \ref{fig:tokamak} 操作相同,只需更改文件名。

图 \ref{fig:tbm_layer} 为插入的pdf格式的矢量图,图 \ref{fig:confusion} 为插入的eps格式的矢量图。一些简单的示意图可以用PowerPoint制作,最后导出成pdf即可,值得注意的是,MS Office套件由于自身的漏洞,无法导出eps格式的文件。
\begin{figure}[H] % H表示强制图片位置,配合float宏包使用,模板中已配置
	\centering
	\includegraphics[width=1\textwidth]{image/chap03/tbm_layer.pdf}
	\caption{插入的pdf格式矢量图}
	\label{fig:tbm_layer}
\end{figure}
\begin{figure}[h]
	\centering
	\includegraphics[width=0.6\textwidth]{image/chap03/confusion.eps}
	\caption{插入的eps格式矢量图}
	\label{fig:confusion}
\end{figure}
\section{多张图片的插入}
\label{sec:fig_multifig}
多张图片插入的原则与单张图片的相同,但是值得注意的是,多张图片不宜使用\LaTeX\ 直接插入,应将所需插入的图片先用PowerPoint排列、拼接,再标号,生成一张图片,再整个插入论文中,这样就与单张图片的插入过程相同。生成图片的过程,偷懒的话可以直接截屏保存为png格式图片,不偷懒就调整ppt的大小后直接导出为pdf。
\begin{figure}[h] 
	\centering
		\includegraphics[width=1\textwidth]{image/chap03/temperature_density.png}
		\caption{简单的两张图片插入。(a)温度分布;(b)密度分布}
		\label{fig:temperature_density}
\end{figure}
\begin{figure}[h]
	\centering
	\includegraphics[width=1\textwidth]{image/chap03/compare.png}
	\caption{多张图片并排插入。(a)图像;(b)真值;(c) CNN+5LSTM1;(d) CNN+5LSTM2;\\ (e) CNN+5LSTM3;(f) CNN+5LSTM4;(g) CNN+5LSTM5}
	\label{fig:compare}
\end{figure}

如果实在有需要直接插入多张图片的需求,则参考如图 \ref{fig:section_compare} 所示的例子。
\begin{figure}[H]
	\begin{subfigure}{0.5\textwidth}
		\centering
		\includegraphics[width=1\textwidth]{image/chap03/section_compare_ddn.png}
	\end{subfigure}
	\begin{subfigure}{0.5\textwidth}
		\centering
		\includegraphics[width=1\textwidth]{image/chap03/section_compare_ddtp.png}
	\end{subfigure}
	\\
	\begin{subfigure}{1\textwidth}
		\centering
		\includegraphics[width=0.5\textwidth]{image/chap03/section_compare_dt.png}
	\end{subfigure}
	\caption{R-matrix理论与ENDF/B-VII.1数据库对比。\\ (a) D(d, n)$^{\text{3}}$He;(b) D(d, p)T;(c) T(d, n)$ ^{\text{4}} $He}
	\label{fig:section_compare}
\end{figure}
\section{本章小结}

\chapter{公式、表格与代码的插入示例}
\label{cha:for_tab_example}
公式用于对论文基础理论的介绍,表格则是对一些不方便进行作图的数据进行展示。

\section{公式的插入}
\label{sec:formula}
带左半边大括号的核反应方程式,如式(\ref{eqn:fusion_reactions})所示:
\begin{equation}
	\label{eqn:fusion_reactions}
	\left\{
	\begin{aligned}
		&\mbox{D}+\mbox{D}\rightarrow \mbox{T}\,(\text{1.01}\;\mbox{MeV})+\mbox{p}\,(\text{3.03}\;\mbox{MeV}) \\
		&\mbox{D}+\mbox{D}\rightarrow {^{\text{3}}}\mbox{He}\,(\text{0.82}\;\mbox{MeV})+\mbox{n}\,(\text{2.45}\;\mbox{MeV}) \\
		&\mbox{D}+\mbox{T}\rightarrow \text{α}\,(\text{3.52}\;\mbox{MeV})+\mbox{n}\,(\text{14.06}\;\mbox{MeV}) \\
		&\mbox{D}+{^{\text{3}}}\mbox{He}\rightarrow \text{α}\,(\text{3.67}\;\mbox{MeV})+\mbox{p}\,(\text{14.67}\;\mbox{MeV})
	\end{aligned}
	\right.
\end{equation}

狄拉克函数$\delta_{ij}$的表达式:
\begin{equation}
	\label{eqn:delta_ij}
	\delta_{ij}=\left\{
	\begin{aligned}
		1,\; i=j \\
		0,\; i\neq j
	\end{aligned}
	\right.
\end{equation}

一般的公式:
\begin{equation}
	\label{eqn:vec_v_cm}
	\vec{v}_{cm}=\dfrac{m_{1}\vec{v}_{1}+m_{2}\vec{v}_{2}}{m_{1}+m_{2}}
\end{equation}

超长的公式\cite{appelbe2011production}:
\begin{equation}
	\label{eqn:iint_theta3_phi3}
	\begin{split}
		\int_{0}^{\pi}\int_{0}^{2\pi} \sin\theta_{3}&\dfrac{\exp(-\alpha v_{cm}^{2})}{v_{cm}}\sinh(\mu \gamma v_{r}v_{cm})\mbox{d}\phi_{3}\mbox{d}\theta_{3}=\dfrac{2\pi \sqrt{\pi}}{4\sqrt{\alpha}v_{3}u_{3}}\exp\left( \dfrac{(\mu \gamma v_{r})^{2}}{4\alpha} \right) \\
		&\times \Bigg( \mbox{erf}\left( \dfrac{\mu \gamma v_{r}+2\alpha(v_{3}-u_{3})}{2\sqrt{\alpha}} \right)-\mbox{erf}\left( \dfrac{-\mu \gamma v_{r}+2\alpha(v_{3}-u_{3})}{2\sqrt{\alpha}} \right) \\
		&+\mbox{erf}\left( \dfrac{-\mu \gamma v_{r}+2\alpha(v_{3}+u_{3})}{2\sqrt{\alpha}} \right)-\mbox{erf}\left( \dfrac{\mu \gamma v_{r}+2\alpha(v_{3}+u_{3})}{2\sqrt{\alpha}} \right) \Bigg)
	\end{split}
\end{equation}

输入矩阵:
\begin{equation}
	\label{eqn:matrix}
	A_{m\times n}=
	\left[ {\begin{array}{cccc}
			a_{11} & a_{12} & \cdots & a_{1n}\\
			a_{21} & a_{22} & \cdots & a_{2n}\\
			\vdots & \vdots & \ddots & \vdots\\
			a_{m1} & a_{m2} & \cdots & a_{mn}\\
	\end{array}}\right]
\end{equation}

\section{表格的插入}
\label{sec:table}
插入一般的表格:
\input{table/chap04_siftflow}

较为复杂的表格:
\begin{table}[h]
	\renewcommand\arraystretch{1.5}
	\centering
	\caption{较为复杂的表格(涉及单元格的合并与拆分)}
	\begin{tabular}{*{5}{c}}
		\toprule
		区域 & \tabincell{c}{外侧核热功率\\(MW)} & \tabincell{c}{内侧核热功率\\(MW)} & 结构 & \tabincell{c}{结构核热功率\\(MW)} \\
		\midrule
		第一壁涂层 & 20.0 & 13.4 & \multirow{2}{*}{第一壁} & \multirow{2}{*}{151.7} \\
		第一壁结构层 & 70.2 & 48.1 & ~ & ~ \\
		\midrule
		Be-1区 & 37.9 & 26.5 & \multirow{4}{*}{氚增殖区} & \multirow{4}{*}{736.2} \\ 
		Li$ _{\text{4}} $SiO$ _{\text{4}} $-1区 & 126.7 & 86.8 & ~ & ~ \\
		Be-2区 & 133.6 & 94.1 & ~ & ~ \\
		Li$ _{\text{4}} $SiO$ _{\text{4}} $-2区 & 134.4 & 96.2 & ~ & ~ \\
		\bottomrule
	\end{tabular}
	\label{tab:chap05_nucheat_tot}
\end{table}

\section{代码的插入}
\label{sec:code}
本模版支持在论文中插入代码片段,或直接从源码文件进行插入。例如,在论文中插入代码片段的效果为:
\begin{python}
def func():
print("hello world")
with open('./output.txt', 'w') as f:
L = f.readlines()
	
if __name__ == "__main__":
# this is a comment line
func()
\end{python}
也可在行内插入代码片段,例如:Python中重载加法运算符的函数为\pyinline{__add__},类的标识符为\pyinline{class}。
此外,还可直接插入代码文件,例如插入\texttt{./code/demo.cpp}的效果为:
\lstinputlisting[style=sysucpp]{code/demo.cpp}
\chapter{结论与展望}
\label{cha:conclusion_outlook}
%% 第六章,有需要可自行使用
%\include{docs/chap07}

% 附录部分
\backmatter
% 参考文献。因不需要纳入章节目录,故放入附录部分
% 实际上参考文献是属于论文主体部分
\makereferences
% 论文定稿打印时使用
\begin{publications}{99}
\setlength{\itemsep}{0em}

\item \textbf{Qiu H}, Lai J, Huang J, et al. Semi-supervised discriminant analysis based on UDP regularization[C]//2008 19th International Conference on Pattern Recognition. IEEE, 2008: 1-4.

\item \textbf{Qiu H}, Liu W, Lai J H. Gender recognition via locality preserving tensor analysis on face images[C]//Asian Conference on Computer Vision. Springer, Berlin, Heidelberg, 2009: 601-610.

\item \textbf{Qiu H}, Pham D S, Venkatesh S, et al. A fast extension for sparse representation on robust face recognition[C]//2010 20th International Conference on Pattern Recognition. IEEE, 2010: 1023-1027.

\item \textbf{Qiu H}, Pham D S, Venkatesh S, et al. Innovative sparse representation algorithms for robust face recognition[J]. International journal of innovative computing, information \& control, 2011, 7(10): 5645-5667.

\item \textbf{Qiu H}, Chen X, Liu W, et al. A fast $\ell_1$-solver and its applications to robust face recognition[J]. Journal of Industrial and Management Optimization (JIMO), 2012, 8: 163-178.

\end{publications}
% 附录
\begin{appendix}
    \chapter{补充材料}


    \chapter{MCNP输入卡}

\begin{python}
1D model of CFETR core                                                          
c cell card for CFETR core                                                     
1 0 -1 25 -26
2 6 -5.75 1 -2 25 -26
3 0 2 -3 25 -26
4 4 -8.017 3 -4 25 -26
5 0 4 -5 25 -26
6 3 -7.709 5 -6 25 -26
7 0 6 -7 25 -26
8 2 -4.515 7 -8 25 -26
9 1 -3.433 8 -9 25 -26
10 0 9 -10 25 -26                    

c surface card
1 CZ 115
2 CZ 225
3 CZ 240.5
4 CZ 348                                            
5 CZ 353
6 CZ 357
7 CZ 362
8 CZ 392
9 CZ 486
10 CZ 500                                                                    

mode n
c importance card
imp:n 16 16 8 8 4 4 2 2 2 1 1 1 2 2 4 4 8 8 16 32 64 64 150 150 0
\end{python}

\end{appendix}
% 致谢
%%
% 致谢
% 谢辞应以简短的文字对课题研究与论文撰写过程中曾直接给予帮助的人员(例如指导教师、答疑教师及其他人员)表示对自己的谢意,这不仅是一种礼貌,也是对他人劳动的尊重,是治学者应当遵循的学术规范。内容限一页。

\chapter{致谢}

在我的硕士论文完成之际,首先感谢我的导师大明副教授以及同课题组的C老师。两位老师时常监督论文进度,并对论文的结构与逻辑提出很多宝贵意见,为论文按时、有质量地完成起了关键作用。感谢我的父母和家人,在中法核的七年中,我的每一个重要决定都有着他们的支持,让我能一步一步坚定地走下去。感谢同课题组的Z师姐和W师弟,在课题组一年,带我下的馆子数量比我前六年的总和都多。

感谢Microsoft旗下的GitHub为本论文的代码提供托管服务;感谢中山大学中法核的X同学以及585所的工程师L提供了本论文所需要的数据以及对计算方法进行了一定指导;感谢中山大学超算队为论文提供了\LaTeX\ 模板;感谢法国CEA的Y同学对\LaTeX\ 使用上的指导;感谢728所的工程师D提供了Notability的使用权;感谢我的朋友们,一路走来,有友情相伴,生活便多了几分色彩。感谢Diana,粉红色的小羽毛球每次都会提醒我要好好吃饭;感谢Eileen,让我知道生活再忙再累,也要抽空去吃自己喜欢的火锅,也告诉我某些东西对我来说是一颗糖,应该是锦上添花的那种,而不是戒断反应的那种;感谢Bella,告诉我要好好锻炼身体,也告诉我勇敢牛牛,不怕困难,干就完事了;感谢Ava,告诉我就算是一只随波漂流的水母,也要拥有自己的梦想;感谢Carol,告诉我要珍惜眼前人,否则拥有再多的骑士也找不回失去的公主。

短短致谢,不能表达对每一个人的感谢,正如短短论文,不能代表在中法核的七年,也如中法核七年,不能代表我的人生。凡是过往,皆为序章,望自己在未来的生活中能牢牢把握前进的方向,不忘初心。


\end{document}
