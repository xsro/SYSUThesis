\chapter{浮动体}
\label{cha:float}
本章将介绍本模板的浮动体,包括表格、插图以及算法环境。
\section{三线表}
\label{sec:sheet}
三线表是论文写作中常见的表格形式,一般的表格如表\ref{tab:siftflow}所示:
\begin{table}[h] %voc table result
	\renewcommand\arraystretch{1.5}
	\centering
		\caption{典型的实验对比表格}	
		\begin{tabular}{*{4}{c}}
			\toprule
	 		Method & Pixel Acc. & Mean Acc. & Mean Iu.\\
			\midrule
			Liu等人\cite{liu2011sift}  & 76.7 & - & -\\
			Tighe等人\cite{tighe2013finding}  & 78.6 & 39.2 & -\\
			FCN-16s\cite{long2015fully} & 85.2 & 51.7 & 39.5\\
			Deeplab-LargeFOV\cite{chen14semantic} & 85.6 & 51.2 & 39.7\\
			\midrule
			Grid-LSTM5 & 86.2 & 51.0 & 41.2\\
			\bottomrule
		\end{tabular}	
		\label{tab:siftflow}
\end{table}


较为复杂的表格:
\input{table/chap02_nucheat_tot}

编制表格应简单明了,表达一致,明晰易懂,表文呼应、内容一致。排版时表格字号略小,或变换字体,尽量不分页,尽量不跨节。表格太大需要转页时,需要在续表上方注明“续表”,表头页应重复排出。

\section{插图}
\label{sec:figure}
有些人可能听说“\LaTeX 只能使用 eps 格式的图片”,甚至把 jpg 格式转为 eps。事实上,这种做法已经过时。而且每次编译时都要要调用外部工具解析 eps,导致降低编译速度。所以我们推荐矢量图直接使用 pdf 格式,位图使用 jpeg 或 png 格式。
\begin{figure}[h] 
	\centering
	\includegraphics[width=0.45\textwidth]{image/chap02/confusion.pdf}
	\caption{图片插入示例}
	\label{fig:confusion}
\end{figure}

关于图片的并排,推荐使用较新的 subcaption 宏包,不建议使用 subfigure 或 subfig 等宏包,但对于多张图片还是不宜使用\LaTeX 直接插入,应将所需插入的图片先用PowerPoint排列、拼接,再标号,生成一张图片,再整个插入论文中。

\section{算法环境}
\label{sec:algorithm}
模板中使用 algorithm2e 宏包实现算法环境。关于该宏包的具体用法,请阅读宏包的官方文档。

\normalem
\begin{algorithm}[h]
	\SetAlgoLined
	\KwData{this text}
	\KwResult{how to write algorithm with \LaTeX2e }
  
	initialization\;
	\While{not at end of this document}{
	  read current\;
	  \eIf{understand}{
		go to next section\;
		current section becomes this one\;
	  }{
		go back to the beginning of current section\;
	  }
	}
	\caption{算法示例1}
	\label{algo:algorithm1}
  \end{algorithm}
  \ULforem

  注意,我们可以在论文中插入算法,但是插入大段的代码是愚蠢的。